\documentclass{article}
\usepackage{amsmath}
\usepackage{amssymb}
\usepackage{graphicx}


\title{\textbf{CSCI 2610 Homework 4}}
\date{\textbf{October 18th, 2024}}
\author{by Swapnil Roy}

\begin{document}

\maketitle
\raggedright

% Main section title
\section*{Section 2.3}

% Sub-title for Question 10
\subsection*{Question 22}
\begin{itemize}
    \item[(a)] Yes. It is injective since it is a linear function that passes the horizontal line test.  It is surjective because the range of a linear function is all real numbers.
    
    \item[(b)] No. It is not injective because it is a quadratic function which fails the horizontal line test. It is not surjective because the range of a quadratic function with a negative leading coefficient is bounded to a range of [-7,7]
    
    \item[(c)] No. While yes, it is injective because solving \( \frac{x+1}{x+2} = \frac{y+1}{y+2} \) leads to \( x = y \), producing a linear function that passes the vertical line test. However, 
    it is not surjective because the function has a vertical asymptote at \( x = -2 \), meaning that there are some invalid inputs. Not being surjective means no bijection.

    \item[(d)] Yes. The function \( f(x) = x^5 + 1 \) is a bijection because it passes the hoirzontal line test. It is surjective because its range covers all real numbers.
\end{itemize}



% Sub-title for Question 30
\subsection*{Question 30}
\begin{itemize}
    \item[(a)] \( f(S) = \{1\} \)
    \item[(b)] \( f(S) = \{-1, 1, 5, 9, 15\} \)
    \item[(c)] \( f(S) = \{0, 1, 2\} \)
    \item[(d)] \( f(S) = \{1, 2, 6, 17\} \)
\end{itemize}

% Main section title
\section*{Section 2.4}

% Sub-title for Question 16
\subsection*{Question 16}
\begin{itemize}
    \item[(a)] \( Solution: a_n = (-1)^n \cdot 5 \)
    \[
    \begin{aligned}
    a_1 &= -a_0 = -5 \\
    a_2 &= -a_1 = 5 \\
    a_3 &= -a_2 = -5 \\
    \end{aligned}
    \]

    \item[(b)] \( Solution: a_n = 1 + 3n \)
    \[
    \begin{aligned}
    a_1 &= a_0 + 3 = 1 + 3 = 4 \\
    a_2 &= a_1 + 3 = 4 + 3 = 7 \\
    a_3 &= a_2 + 3 = 7 + 3 = 10 \\
    \end{aligned}
    \]

    \item[(c)] \( Solution: a_n = 4 - \frac{n(n+1)}{2} \)
    \[
    \begin{aligned}
    a_1 &= a_0 - 1 = 4 - 1 = 3 \\
    a_2 &= a_1 - 2 = 3 - 2 = 1 \\
    a_3 &= a_2 - 3 = 1 - 3 = -2 \\
    \end{aligned}
    \]

    \item[(d)] \( Solution: a_n = 2a_{n-1} - 3 \)
    \[
    \begin{aligned}
    a_1 &= 2a_0 - 3 = 2(-1) - 3 = -5 \\
    a_2 &= 2a_1 - 3 = 2(-5) - 3 = -13 \\
    a_3 &= 2a_2 - 3 = 2(-13) - 3 = -29 \\
    \end{aligned}
    \]
\end{itemize}

% Sub-title for Question 32
\subsection*{Question 32}
\begin{itemize}
    \item[(a)] \( 10 \)
    \item[(b)] \( 9330 \)
    \item[(c)] \(21215 \)
    \item[(d)] \(511 \)
\end{itemize}


% Main section title
\section*{Section 2.5}
% Sub-title for Question 4
\subsection*{Question 4}
\begin{itemize}
    \item[(a)] countable.
    \item[(b)] countable.
    \item[(c)] countable.
    \item[(d)] uncountable.
\end{itemize}

% Sub-title for Question 6
\subsection*{Question 6}
\begin{itemize}
    \item[1:] The guests in odd-numbered rooms stay in their current rooms.
    
    \item[2:] For each guest in an even-numbered room, move them to an available odd-numbered room.

    \item[3:] Since there are infinitely many odd-numbered rooms and the number of guests is countably infinite, this mapping is a bijection. Each guest from an even-numbered room is moved to an odd numbered room a countably finite number of rooms away, leaving no guest without a room.
    
    \item[4:] All guests are now accommodated in the available odd-numbered rooms, and no guest is left without a room.
\end{itemize}

\textbf{Conclusion:} All guests can remain in the hotel by moving guests from even-numbered rooms to the available odd-numbered rooms.


% Main section title 12.1
\section*{Section 12.1}
\subsection*{Question 2.a}
\[
\begin{array}{|c|c|c|}
\hline
x & y & \overline{x} + y \\
\hline
0 & 0 & 1 \\
0 & 1 & 1 \\
1 & 0 & 0 \\
1 & 1 & 1 \\
\hline
\end{array}
\]

\[
F(x, y) = \overline{x} \overline{y} + \overline{x} y + x y
\]

\subsection*{Question 2.b}
\[
\begin{array}{|c|c|c|}
\hline
x & y & x \overline{y} \\
\hline
0 & 0 & 0 \\
0 & 1 & 0 \\
1 & 0 & 1 \\
1 & 1 & 0 \\
\hline
\end{array}
\]
\[
F(x, y) = x \overline{y}
\]

\subsection*{Question 2.c}
\[
F(x, y) = 1
\]

\subsection*{Question 2.d}
\[
\begin{array}{|c|c|c|}
\hline
x & y & \overline{y} \\
\hline
0 & 0 & 1 \\
0 & 1 & 0 \\
1 & 0 & 1 \\
1 & 1 & 0 \\
\hline
\end{array}
\]
\[
F(x, y) = \overline{x} \overline{y} + x \overline{y}
\]

% Main section title 12.2
\section*{Section 12.2 (last page)s}

\subsection*{Question 2}
\begin{figure}[h!]
\centering
\includegraphics[width=\textwidth]{logic-gates.jpg}
\caption{Logic gates diagram. Problem 2}
\label{fig:logic-gates}
\end{figure}

% Main section title 12.3
\section*{Section 12.3 (last pages)}

\subsection*{Question 6}
\begin{figure}[h!]
\centering
\includegraphics[width=\textwidth]{kmap.jpg}
\caption{Kmap. Problem 6 a and b}
\label{fig:kmap}
\end{figure}

\subsection*{Question 12}
\begin{figure}[h!]
\centering
\includegraphics[width=\textwidth]{kmap2.jpg}
\caption{Kmap. Problem 12 a-d}
\label{fig:kmap2}
\end{figure}
\end{document}
