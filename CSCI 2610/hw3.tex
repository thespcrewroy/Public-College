\documentclass{article}
\usepackage{amsmath}
\usepackage{amssymb}

\title{\textbf{CSCI 2610 Homework 3}}
\date{\textbf{October 4th, 2024}}
\author{by Swapnil Roy}

\begin{document}

\maketitle
\raggedright

% Main section title
\section*{Section 2.1}

% Sub-title for Question 10
\subsection*{Question 10}
% Answers to Question 10:
\begin{itemize}
    \item[(a)] No
    \item[(b)] No
    \item[(c)] Yes
    \item[(d)] Yes
    \item[(e)] Yes
    \item[(f)] No
\end{itemize}

% Sub-title for Question 12
\subsection*{Question 12}
% Answers to Question 12:
\begin{itemize}
    \item[(a)] True
    \item[(b)] True
    \item[(c)] False
    \item[(d)] True
    \item[(e)] True
    \item[(f)] False
    \item[(g)] False
\end{itemize}

% Sub-title for Question 20
\subsection*{Question 20}
% Answers to Question 20
ANS) \( A = \emptyset \) and \( B = \{\emptyset\} \).


% Sub-title for Question 26
\subsection*{Question 26}
% Answers to Question 26
\begin{itemize}
    \item[(a)] No
    \item[(b)] Yes
    \item[(c)] No
    \item[(d)] Yes
\end{itemize}


% Sub-title for Question 34
\subsection*{Question 34}
% Answers to Question 34
\begin{itemize}
    \item[(a)] \( A \times B \times C = \)
    \[
    \begin{aligned}
    \{ (a, x, 0), & (a, x, 1), (a, y, 0), (a, y, 1), \\
    (b, x, 0), & (b, x, 1), (b, y, 0), (b, y, 1), \\
    (c, x, 0), & (c, x, 1), (c, y, 0), (c, y, 1) \}
    \end{aligned}
    \]
    
    \item[(b)] \( C \times B \times A = \)
    \[
    \begin{aligned}
    \{ (0, x, a), & (0, x, b), (0, x, c), (0, y, a), (0, y, b), (0, y, c), \\
    (1, x, a), & (1, x, b), (1, x, c), (1, y, a), (1, y, b), (1, y, c) \}
    \end{aligned}
    \]
    
    \item[(c)] \( C \times A \times B = \)
    \[
    \begin{aligned}
    \{ (0, a, x), & (0, a, y), (0, b, x), (0, b, y), (0, c, x), (0, c, y), \\
    (1, a, x), & (1, a, y), (1, b, x), (1, b, y), (1, c, x), (1, c, y) \}
    \end{aligned}
    \]
    
    \item[(d)] \( B \times B \times B = \)
    \[
    \begin{aligned}
    \{ (x, x, x), & (x, x, y), (x, y, x), (x, y, y), \\
    (y, x, x), & (y, x, y), (y, y, x), (y, y, y) \}
    \end{aligned}
    \]
\end{itemize}

% Main section title
\section*{Section 2.2}
% Sub-title for Question 4
\subsection*{Question 4}
% Answers to Question 4:
\begin{itemize}
    \item[(a)] $\{a, b, c, d, e, f, g, h\}$
    \item[(b)] $\{a, b, c, d, e\}$
    \item[(c)] $\emptyset$
    \item[(d)] $\{f, g, h\}$
\end{itemize}

% Sub-title for Question (a)
\subsection*{Question 8.a}
To prove it, we need to prove $A \cup A \subseteq A$ and $A \subseteq A \cup A$.

\textbf{Proof: $A \cup A \subseteq A$.}
\begin{itemize}
    \item[1.] \( x \in A \cup A \) \hfill \textit{Assumption}
    \item[2.] \( x \in A \lor x \in A \) \hfill \textit{Definition of Union}
    \item[3.] \( x \in A \) \hfill \textit{Idempotent Law of Propositions}
    \item[4.] Therefore, $A \cup A \subseteq A$
\end{itemize}

\textbf{Proof: $A \subseteq A \cup A$.}
\begin{itemize}
    \item[1.] \( x \in A \) \hfill \textit{Assumption}
    \item[2.] \( x \in A \lor x \in A \) \hfill \textit{Definition of Union}
    \item[3.] \( x \in A \cup A \) \hfill \textit{Definition of Union}
    \item[4.] Therefore, $A \subseteq A \cup A$
\end{itemize}

\textbf{Since $A \cup A \subseteq A$ and $A \subseteq A \cup A$, $A \cup A = A$ is true.}

% Sub-title for Question (b)
\subsection*{Question 8.b}
To prove it, we need to prove $A \cap A \subseteq A$ and $A \subseteq A \cap A$.

\textbf{Proof: $A \cap A \subseteq A$.}
\begin{itemize}
    \item[1.] \( x \in A \cap A \) \hfill \textit{Assumption}
    \item[2.] \( x \in A \land x \in A \) \hfill \textit{Definition of Intersection}
    \item[3.] \( x \in A \) \hfill \textit{Idempotent Law of Propositions}
    \item[4.] Therefore, $A \cap A \subseteq A$
\end{itemize}

\textbf{Proof: $A \subseteq A \cap A$.}
\begin{itemize}
    \item[1.] \( x \in A \) \hfill \textit{Assumption}
    \item[2.] \( x \in A \land x \in A \) \hfill \textit{Definition of Intersection}
    \item[3.] \( x \in A \cap A \) \hfill \textit{Definition of Intersection}
    \item[4.] Therefore, $A \subseteq A \cap A$
\end{itemize}

\textbf{Since $A \cap A \subseteq A$ and $A \subseteq A \cap A$, $A \cap A = A$ is true.}

% Sub-title for Question 14
\subsection*{Question 14}
\begin{itemize}
    \item[] \( A = \{1, 3, 5, 6, 7, 8, 9\} \)
    \item[] \( B = \{2, 3, 6, 9, 10\} \)
\end{itemize}

% Sub-title for Question 34
\subsection*{Question 34}
\begin{itemize}
    \item[1.] \( (A - B) \cap (B - C) \cap (A - C) \) \hfill \textit{Given}
    \item[2.] \( (A - B) \cap (B - C) \cap (A - C) \) \hfill \textit{Definition of Set Difference}
    \item[3.] \((A \cap \overline{B}) \cap (B \cap \overline{C}) \cap (A \cap \overline{C}) \) \hfill \textit{Associative Law}
    \item[4.] \( = A \cap \overline{B} \cap B \cap \overline{C} \cap A \cap \overline{C} \) \hfill \textit{Complement Law \#2}
    \item[5.] \( = \emptyset \) \hfill \textit{Domination Law \#2}
\end{itemize}

\text{Therefore, \( (A - B) \cap (B - C) \cap (A - C) = \emptyset \) is true}

% Sub-title for Question 54
\subsection*{Question 54.a}
\[
\bigcup_{i=1}^n A_i = \bigcup_{i=1}^n \{\ldots, -2, -1, 0, 1, \ldots, i\} = \mathbb{Z}
\]

\subsection*{Question 54.b}
\[
\bigcap_{i=1}^n A_i = \bigcap_{i=1}^n \{\ldots, -2, -1, 0, 1, \ldots, i\} = \{\ldots, -2, -1, 0\}
\]
\end{document}